%% Generated by Sphinx.
\def\sphinxdocclass{report}
\documentclass[letterpaper,10pt,english]{sphinxmanual}
\ifdefined\pdfpxdimen
   \let\sphinxpxdimen\pdfpxdimen\else\newdimen\sphinxpxdimen
\fi \sphinxpxdimen=49336sp\relax

\usepackage[margin=1in,marginparwidth=0.5in]{geometry}
\usepackage[utf8]{inputenc}
\ifdefined\DeclareUnicodeCharacter
  \DeclareUnicodeCharacter{00A0}{\nobreakspace}
\fi
\usepackage{cmap}
\usepackage[T1]{fontenc}
\usepackage{amsmath,amssymb,amstext}
\usepackage{babel}
\usepackage{times}
\usepackage[Bjarne]{fncychap}
\usepackage{longtable}
\usepackage{sphinx}

\usepackage{multirow}
\usepackage{eqparbox}

% Include hyperref last.
\usepackage{hyperref}
% Fix anchor placement for figures with captions.
\usepackage{hypcap}% it must be loaded after hyperref.
% Set up styles of URL: it should be placed after hyperref.
\urlstyle{same}
\addto\captionsenglish{\renewcommand{\contentsname}{Table of Contents:}}

\addto\captionsenglish{\renewcommand{\figurename}{Fig.\@ }}
\addto\captionsenglish{\renewcommand{\tablename}{Table }}
\addto\captionsenglish{\renewcommand{\literalblockname}{Listing }}

\addto\extrasenglish{\def\pageautorefname{page}}

\setcounter{tocdepth}{1}



\title{Suzaku SNR Catalog Documentation}
\date{Mar 31, 2017}
\release{}
\author{Nikolai Shaposhnikov, Lorella Angelini}
\newcommand{\sphinxlogo}{}
\renewcommand{\releasename}{Release}
\makeindex

\begin{document}

\maketitle
\sphinxtableofcontents
\phantomsection\label{\detokenize{index::doc}}



\chapter{Introduction}
\label{\detokenize{intro:welcome-to-suzaku-snr-catalog-s-documentation}}\label{\detokenize{intro:introduction}}\label{\detokenize{intro::doc}}
The Suzaku SNR Catalog is provides a comprehesive analysis of Suzaku XIS data on
supernova remnant (SNR) sources. An attempt is made to provide a coherent data products
and analysis for the entire set of Suzaku observations of SNRs. Catalog provides
a web interface to access the data products and the analysis results. The list of
the detected emission line list is also provided as the HEASARC table, where the
basic line properites are accompanied by additional information on the source, i.e its
type, presence of a pulsar, source coverage, observation region, etc.

The analysis product includes images, spectra and spectral analysis.
Spectra are modeled as a sum of power law and a set gaussians.
The results of spectral modeling are provided as table of individual spectral lines.
Currently catalog contains analysis of 615 Suzaku observations of 161 sources.

Each observation has one or more analysis results, performed
for a particular region nselction.  An individual page is provided
for each analysis, which features line list for this analysis
and spectral plots.

In the next two setions we describe the source selection and Suzaku observation
analysis procedure for the catalog
in more detail.


\chapter{Source Selection}
\label{\detokenize{sources:source-selection}}\label{\detokenize{sources::doc}}
The primary source for the catalog object list is  the Green's SNR catalog %
\begin{footnote}[1]\sphinxAtStartFootnote
\url{https://heasarc.gsfc.nasa.gov/W3Browse/all/snrgreen.html}
%
\end{footnote},
which includes 294 individual SNR objects. The Green's catalog
is then  augmented by the LMC and SMC SNRs as provided by the SIMBAD astronomical
source database %
\begin{footnote}[2]\sphinxAtStartFootnote
\url{http://simbad.u-strasbg.fr/simbad/}
%
\end{footnote}. In addition, the source list contains two sources G023.5+00.1
(galactic SNR not included in the Green's catalog) and SN 2014J (extragalactic SN).

To construct the SNR lists for LMC and SMC we querried the SIMBAD database
to yeild the SNR sources (otype parameter is set to `SNR') within the circlular
region centered on coordinates RA=05 23 34.6 DEC=-69 45 22 with radius of 11
degrees for  the LMC and RA=00 52 38.0, DEC -72 48 01 with radius of
5.5 degrees for SMC. This procedure yielded 118 and 35 individual
sources  for SMC and LMC correspondingly.

Therefore, the initial list used for the Suzaku observation selection contained
449 objects. This initial list was then cross-correlated with the Suzaku Master Table %
\begin{footnote}[3]\sphinxAtStartFootnote
\url{http://heasarc.gsfc.nasa.gov/w3browse/all/suzamaster.html}
%
\end{footnote}
to get the list of object observed by Suzaku.

The process of the observations selection is described
in the next section.


\chapter{Observation Selection}
\label{\detokenize{observations:observation-selection}}\label{\detokenize{observations::doc}}
Initial list of observations for the catalog analysis included all the sources
which field of view covered a part of a source either fully or partially.
This was done by searching the entire set Suzaku observations, creating
XIS field of view region and determining if it intersects with any of
the SNR source regions. This search yielded 615 individual observation sequences.
The set of spectra and images was produced for all SNR observations in this
list. The data reduction procedure and data products are described in the
``Analysis'' section.

Further downselection were made to filter out observations irrelevant
for the spectral analysis of the emission lines from SNRs.
Namely, the data from featureless
sources like the Crab or G21.5-0.9, observations dominated by unrelated
sources or observations having strongly variable X-ray background
with the narrow lines (i. e. the observations near the Galactic Center and
Galactic Bulge).


\chapter{Data Reduction Procedure}
\label{\detokenize{procedure:data-reduction-procedure}}\label{\detokenize{procedure::doc}}
To extract data products for the catalog we developed an automated analysis pipeline, implemented as PERL script aexispipe. The particular analysis procedure is controlled by the script parameters. Where appropriate, we give the corresponding analysis options, which responsible for a certain analysis setting. We also list the analysis script settings in a separate section below. The presented analysis attempts to perform XISdata analysis as described in the Suzaku Data Reduction Guide (aka `'ABC Guide'`). Within this pipeline we have implemented the following analysis steps:

1. Observation sequence data download via FTP and initial reprocessing with the Suzaku FTOOL aepipeline.

2. Clean event list production. This step involves filtering out events recorded by flickering pixels. Regions, containing information on the flickering pixels for particular XIS and CI mode are downloaded from JAXA page at \url{http://www.astro.isas.jaxa.jp/suzaku/analysis/xis/nxb\_new/} and applied to the reprocessed cleaned event files produced at the previous step. 

3. Creating and smoothing  XIS image.

4. Generating the spectral extraction regions.5. Extract spectra for X-ray source and background regions using xselect.

6. Extract spectra and images of the XIS instrumental background commonly referred to as a Non X-ray Background (NXB) using xisnxbgen. NXB event files have also flickerring pixels removed using the region filtering. For convenience and to speed up analysis we have produced pre-filtered NXB event files from the files provided in the calibration database and use these prefiltered files during the analysis.

7. Create XIS detector response for the observation and ancillary response (ARF) for the set of regions using xisrmfgen and xissimarfgen correspondingly. Depending on a particular nature of the source we employ three different prescriptions for the ARF calculations, which we refer to as ``UNIFORM'', ``POINT'' and ``IMAGE''. In the first (``UNIFORM'' source) approach, the source is assumed to be uniform over the sky. Actual ray-tracing simulations in xissimarfgen models the 3×106 input photons coming from the sky region extending 20 arc seconds beyond the XIS field of view. In the ``POINT'' prescription, we assume a point source at the specified object sky location and model 4×105 photons. In the last, i.e. ``IMAGE'' setting, the actual XIS source image is supplied as source configuration and 3×106 are simulated.

8. Produce rebinned spectra and responses using the standard binning pattern. Namely, 4096 unbinned channels are rebinned to 2048 channels by binning channels between 700 and 2696 with factor of 2 and channels between 2696 and 4096 with factor of 4. We perform this step using rbnpha and rbnrmf FTOOLs respectively.

9. NXB subtracted spectra are produced by subtracting the source spectra and NXB spectra with mathpha FTOOL.


\chapter{Data Products Description}
\label{\detokenize{products:data-products-description}}\label{\detokenize{products::doc}}
\# Data products and naming conventions

A particular Suzaku observation analysis in the catalog is a set of data products created for a particular set of analysis script (aexispipe) setting. Therefore, there can be more than one analyses for a particular observation. All data products for analysis reside in one directory under the observation directory. The directory name is usually (but not necessarily) a name of the region pattern, for example, analysis products for the ``pie'' regions are placed under \sphinxstyleemphasis{\textless{}OBSID\textgreater{}/pie} directory.

\# File naming convention

The data product files names follow the rules below:

for spectral files -

aeXXXXXXXXXiii\_rNR\_SSS\_rB.pha,

for ancillary response files -

aeXXXXXXXXXiii\_rNR\_SSS.arf,

for response matrix -

aeXXXXXXXXXiii\_SSS\_rB.rmf,

for region files -

aeXXXXXXXXXiii\_SSS\_src.reg,

for region files -

aeXXXXXXXXXiii\_SSS\_src.img,

where:
\begin{itemize}
\item {} 
ae is for Astro-E2, as for the archival files

\item {} 
XXXXXXXXX is the 9 digit observation identifier (sequence number)

\item {} 
NR is the region identifier, where N is the region number in a set between 0 and 8

\item {} 
R is the region pattern identifier, namely, S - for standard regions, P - for pie regions, A - for annuli, R -for SIMBAD regions.

\item {} 
SSS is the source for spectrum. Namely, src - for total spectrum, nxb - for the NXB spectrum, sxb - for the NXB subtracted spectrum.

\item {} 
B is rebinning indicator, i.e. a - original spectrum, b- rebinned spectrum.

\end{itemize}


\chapter{Data Reduction Procedure}
\label{\detokenize{modeling:data-reduction-procedure}}\label{\detokenize{modeling::doc}}
To extract data products for the catalog we developed an automated analysis pipeline, implemented as PERL script aexispipe. The particular analysis procedure is controlled by the script parameters. Where appropriate, we give the corresponding analysis options, which responsible for a certain analysis setting. We also list the analysis script settings in a separate section below. The presented analysis attempts to perform XISdata analysis as described in the Suzaku Data Reduction Guide (aka `'ABC Guide'`). Within this pipeline we have implemented the following analysis steps:

1. Observation sequence data download via FTP and initial reprocessing with the Suzaku FTOOL aepipeline.

2. Clean event list production. This step involves filtering out events recorded by flickering pixels. Regions, containing information on the flickering pixels for particular XIS and CI mode are downloaded from JAXA page at \url{http://www.astro.isas.jaxa.jp/suzaku/analysis/xis/nxb\_new/} and applied to the reprocessed cleaned event files produced at the previous step. 

3. Creating and smoothing  XIS image.

4. Generating the spectral extraction regions.5. Extract spectra for X-ray source and background regions using xselect.

6. Extract spectra and images of the XIS instrumental background commonly referred to as a Non X-ray Background (NXB) using xisnxbgen. NXB event files have also flickerring pixels removed using the region filtering. For convenience and to speed up analysis we have produced pre-filtered NXB event files from the files provided in the calibration database and use these prefiltered files during the analysis.

7. Create XIS detector response for the observation and ancillary response (ARF) for the set of regions using xisrmfgen and xissimarfgen correspondingly. Depending on a particular nature of the source we employ three different prescriptions for the ARF calculations, which we refer to as ``UNIFORM'', ``POINT'' and ``IMAGE''. In the first (``UNIFORM'' source) approach, the source is assumed to be uniform over the sky. Actual ray-tracing simulations in xissimarfgen models the 3×106 input photons coming from the sky region extending 20 arc seconds beyond the XIS field of view. In the ``POINT'' prescription, we assume a point source at the specified object sky location and model 4×105 photons. In the last, i.e. ``IMAGE'' setting, the actual XIS source image is supplied as source configuration and 3×106 are simulated.

8. Produce rebinned spectra and responses using the standard binning pattern. Namely, 4096 unbinned channels are rebinned to 2048 channels by binning channels between 700 and 2696 with factor of 2 and channels between 2696 and 4096 with factor of 4. We perform this step using rbnpha and rbnrmf FTOOLs respectively.

9. NXB subtracted spectra are produced by subtracting the source spectra and NXB spectra with mathpha FTOOL.


\chapter{Search}
\label{\detokenize{index:search}}\begin{itemize}
\item {} 
\DUrole{xref,std,std-ref}{search}

\end{itemize}



\renewcommand{\indexname}{Index}
\printindex
\end{document}